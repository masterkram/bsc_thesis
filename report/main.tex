\documentclass[acmtog, authorversion]{acmart}
\usepackage[htt]{hyphenat}
\usepackage{graphicx}
\usepackage{lipsum}
\usepackage{pgfgantt}
\usepackage{enumitem}
\usepackage{booktabs}
\usepackage{fontawesome5}
\usepackage{listings}
\usepackage{minted}

\newlist{questions}{enumerate}{2}
\setlist[questions,1]{label=\textbf{RQ\arabic*.},ref=RQ\arabic*}
\setlist[questions,2]{label=(\alph*),ref=\thequestionsi(\alph*)}
\graphicspath{{images/}}

\AtBeginDocument{%
  \providecommand\BibTeX{{%
    \normalfont B\kern-0.5em{\scshape i\kern-0.25em b}\kern-0.8em\TeX}}}

\setcopyright{cc}
\copyrightyear{2023}
\acmYear{2023}

\acmConference[TScIT 39]{39$^{th}$ Twente Student Conference on IT}{July 8,
  2023}{Enschede, The Netherlands}


\begin{document}

\title{Satellite to Radar: Sequence to Sequence Learning for precipitation nowcasting}

\author{Mark Bruderer}
\email{m.a.bruderervanblerk@student.utwente.nl}
\affiliation{%
  \institution{University of Twente}
  \streetaddress{P.O. Box 217}
  \city{Enschede}
  \country{The Netherlands}
  \postcode{7500AE}
}

\renewcommand{\shortauthors}{Mark Bruderer}

\begin{abstract}
\section*{Abstract}
The forecasting of rain is a complex problem with centuries of scientific work. The implications of weather for individuals and companies continue to be important. Machine Learning approaches have been shown to outperform state of the art physics based models of weather for short term predictions. We introduce a new type of model \textbf{Sat2Rad}. Our model takes as input multi-spectral satellite data and outputs radar reflectivity at a set of time-steps ranging from 15 to 180 minutes in the future. This model is novel for precipitation nowcasting because it uses several satellite spectral bands instead of using radar data as input.
\end{abstract}

\keywords{Machine Learning, Sequence to Sequence, Radar, Satellite, Storms, Forecasting}

\settopmatter{printacmref=false}

\begin{teaserfigure}
  \includegraphics*[width=\textwidth, trim=0in 0.0in 0in 16.0in]{images/lightning.jpg}
  \caption{A supercell thunderstorm at twilight in SW Oklahoma.}
  \Description{A supercell thunderstorm at twilight in SW Oklahoma.}
  \label{fig:teaser}
\end{teaserfigure}

\maketitle

\section{Introduction} \label{introduction}
Precipitation forecasting is essential to reduce the risk of life threatening situations. Different types of rainfall ranging from mist to heavy rain have a major impact for different societal sectors including agriculture, aviation, outdoor events, and the energy industry.
By having timely and accurate predictions of rainfall which in turn indicate the potential for destructive storms we can prevent injuries, assist companies in predicting energy production and use resources efficiently.
\medskip

%A particularly strong threat is posed by rain storms and thunderstorms. Storms are one of the most destructive weather events in nature, capable of destroying human structures and even lead to loss of life \cite{noaa-national-severe-storms-laboratory-no-date}. Predicting storms is crucial and presents it's own set of challenges.
%\medskip

At present meteorologists are able to successfully predict many instances of precipitation. Techniques that are used in practice range from manual analysis of current weather data (e.g radar or satellite images) to complex physics based simulations of our atmosphere with Numerical Weather Prediction (\textsc{NWP}) models.
Various short term forecasting methods are based on \textit{optical flow}. Optical flow functions in two steps, first cumulonimbus (storm) clouds are identified, and then their movement is tracked to predict the location of precipitation. Thus in this case, the forming and dissipation of clouds known as a \textit{cell-lifecycle} \cite{noaas-national-weather-service-no-date} is not taken into account \cite{prudden2020review}.
\medskip

Machine Learning (\textsc{ML}) approaches have also been developed to predict precipitation.
An improvement of machine learning models over \textsc{NWP} models is that they are much faster to produce predictions, thus ML models are more suitable for real-time or near-real-time predictions, such as required in disaster response and energy management. These short term predictions are referred to as \textit{nowcasts}. According to the universal approximation theorem \cite{cybenko-1989}, deep neural networks have the property of being able to approximate any function provided they have the correct weights, thus it is suggested that machine learning models can incorporate sources of predictability beyond optical flow such as the \textit{cell-lifecycle}.
Other suggested sources of predictability are: the elevation of terrain, convergence lines and the current time among others \cite{prudden2020review}.
\medskip

Thus far most machine learning approaches for precipitation nowcasting have focused on predicting future frames of currently available radar data \cite{shi2017deep, convlstm, rainet}. However taking this approach may eliminate the possibility of learning the \textit{cell-lifecycle}, due to the fact that the model only sees precipitation itself but not the cloud that is causing the precipitation.
\medskip


We propose to use multi-spectral satellite data to learn spatio-temporal mappings between sequences of satellite data and precipitation data in the near future. A well performing model could predict storm clouds when these clouds are still forming. An additional advantage is that contrary to radar data, satellite data is readily available over oceans and remote communities (See figure \ref{fig:radar-availability}) which allows for the prediction of precipitation over these regions.
\medskip

\subsection{Problem Formulation}
We consider precipitation nowcasting as a self-supervised problem.
%In self-supervised tasks, explicit labels are not provided, but rather we can derive labels from the raw data itself, which is often the case with time-series data. Consequently, we can utilize established techniques in supervised learning to address our research problem.
The prediction of labels can be accomplished through two approaches: predicting discrete classes that correspond to different rain intensity intervals, or conducting pixel-level regression to learn the precise values of precipitation.

\subsubsection{Regression Formulation} Consider a dataset \{X, Y\} consisting of pairs of input-output sequences indexed by $i \in \mathbb{N}$,
%where each input sequence represents a temporal sequence of $t$ satellite images with height $H$, width $W$ and $c$ channels.
Let $$X = \{x^{(i)} \in \mathbb{R}^{t \times h \times w \times c}\} \forall i$$ 
where $x^{(i)}$ is a tensor of dimension $t \times h \times w \times c$ representing the sequence of satellite images at position $i$, having $t$ time-steps, $h$ height, $w$ width and $c$ channels.
The set of output sequences is denoted as $$Y = \{y^{(i)} \in \mathbb{R}^{w\times h}\}\forall i$$
where $y^{(i)}$ represents the $i^{th}$ $h \times w$ dimensional tensor with each pixel being a real number collected by the radar reflectivity reading.
Here the width and height are the same as in the input sequence.
The problem is formulated as finding a function f(x). This function must minimize a chosen distance function $\mathcal{D}$ as follows:
Let $\hat{Y} = \{p(x^{(i)})\}\forall i$, representing the predicted outputs for each sequence of satellite images and identical in dimensions to $y^{(i)}$.
find f(x) such that $\mathcal{D}(\hat{Y}, Y)$ is minimized.

\subsubsection[short]{Classification Formulation}
Consider a dataset \{X, Y\} consisting of pairs of input-output sequences indexed by $i \in \mathbb{N}$,
%where each input sequence represents a temporal sequence of $t$ satellite images with height $H$, width $W$ and $c$ channels.
Let $$X = \{x^{(i)} \in \mathbb{R}^{t \times h \times w \times c}\} \forall i$$ 
where $x^{(i)}$ is a tensor of dimension $t \times h \times w \times c$ representing the sequence of satellite images at position $i$, having $t$ time-steps, $h$ height, $w$ width and $c$ channels.
The set of output sequences is denoted as $$Y = \{y^{(i)} \in \mathbb{N}^{w\times h}\}\forall i$$
where $y^{(i)}$ represents the $i^{th}$ h $\times w$ dimensional tensor containing discrete integers mapping to rainfall intensity classes.
The problem is formulated as finding a probability mass function p(x). This function must minimize the Cross Entropy Loss expressed as follows:

$$E = \frac{1}{h+w}\sum_{i=1}^h\sum_{j=1}^w t_{ij} log(p_{ij})$$

Let $\hat{Y} = \{p(x^{(i)})\}\forall i$, representing the predicted outputs for each sequence of satellite images and identical in dimensions to $y^{(i)}$.
find p(x) such that $E(\hat{Y}, Y)$ is minimized.

\subsection{Research Question}
\textbf{RQ}: \textit{How can a deep learning model be trained to predict radar data with multi-spectral satellite data ?}
\medskip

This research question will be answered by looking at the following sub research questions:
\begin{enumerate}
    \item \textit{how must data be preprocessed and aggregated to create a model capable of predicting precipitation based on satellite data?}
    \item \textit{How can the process of training be simplified to be able to experiment with different architectures ?}
    \item \textit{what model architecture performs the best based on established metrics for classification and regression ?}
\end{enumerate}


\section{Contribution}
In this research we focus on testing the hypothesis if a model trained on satellite sequences is capable of forecasting precipitation.
This is unclear as of now because most other studies rely on only radar or a combination of satellite and radar.
In the precipitation nowcasting field of research, there is a symbiosis between the task of video-to-video prediction and precipitation forecasting, models created for one are used in the other and vice-versa.
It is unclear whether these techniques can be successfully applied when using input and output data from different domains. We compare three different model architectures, UNet, ConvLSTM + Attention and ConvLSTM. All of these techniques are analyzed in both pixel-classification and pixel-regression implementations.

\section{Related Works}
In this section we will discuss the existing work in precipitation nowcasting via machine learning. This section is structured by the type of input data used in the approaches, first we list radar based learning and then we discuss satellite based approaches.

\subsection{Radar Based Nowcasting}

Recurrent neural networks (\textsc{RNNs}) have been created to learn temporal relationships in data, therefore they are a natural candidate to the task of learning spatio-temporal patterns of weather. The \textsc{LSTM} architecture was developed by Hochreiter and Schmidhuber \cite{lstm}, to solve the problem of vanishing and exploding gradients in RNNs and is widely used. Taking LSTM as a base and adapting the weights to kernels, \textsc{ConvLSTM} \cite{convlstm} was introduced for the task of precipitation nowcasting. Multiple layers of ConvLSTM are used in this paper to obtain a sequence to sequence architecture. A further improvement of ConvLSTM is \textsc{trajGRU} which was proposed by Shi et al. \cite{shi2017deep} to be able to learn the \textit{location-variant} structure for recurrent connections.
\medskip

Pure convolutional neural networks have also been used to predict precipitation. As demonstrated by Bai et al. and Gering et al. \cite{bai2018empirical, gehring2017convolutional} convolutional neural architectures can outperform recurrent neural networks for a variety of sequence modelling tasks. This is the reason why many works on precipitation nowcasting have opted for pure convolutional networks \cite{rainet,agrawal2019machine}.
\medskip

Due to machine learning models attempting to minimize loss, \textit{blurry predictions} can be produced by models. This can be alleviated by using generative models which sample from the possible futures and do not seek to provide a best average fit. Generative Adversarial Networks have been successfully applied to the task of precipitation nowcasting \cite{Ravuri_2021}.

\subsection{Satellite Based Nowcasting}

In their study Chen et al. built a \cite{precipitationEstimationFromSat} MLP to forecast radar data from satellite data. The researchers used a combination of low earth orbit satellite passive microwave and infrared channels from two different satellites. Their model is developed to predict up to 1.5 hours in the future by recursive predictions of the model.
\medskip

A study that does not predict precipitation but uses lightning as a marker for extreme precipitation was performed by Brodehl et al. \cite{predictionLightning} this study uses a convolutional network to predict lightning events, and contributes the important observation that both the visual and infrared channels are important in differing ways to predict lightning.
\medskip

The approach taken by the researchers of MetNet \cite{sønderby2020metnet} is to combine a convolutional block for spatial downsampling, then a ConvLSTM block for temporal encoding and finally a Axial attention block \cite{vaswani2017attention}. MetNet is able to perform more accurate forecasts than NWP models for up to 8 hours. In this study the input data that is used is both satellite and radar data as well as the elevation, time of the year and latitude and longitude values.

\section{Background}
In this section, we will explore the machine learning techniques utilized
in this study as well as relevant background information regarding the
provided datasets and meterological use thereof.
\medskip

\subsection{Convolutional Neural Networks}

Convolutional neural networks (CNNs) were initially introduced by Yann LeCun et al. in 1998 \cite{lecun-1998} and applied to the challenge of handwritten digit classification. However, the breakthrough for CNNs came later with the success achieved by Krizhevsky et al \cite{krizhevsky-2017}. in the ImageNet paper of 2012.
This work significantly advanced the state of the art of image classification.
\medskip

CNNs are a class of deep learning models strongly capable of solving computer vision tasks. Unlike fully connected neural networks, which treat input data as a one dimensional vector, CNNs are designed to process higher dimensional data such as images.
This distinction enables CNNs to exploit more spatial relationships and patterns in visual data as opposed to a flattened vector where these patters are not recoverable.
\medskip

Additionally Convolutional neural networks reduce the amount of parameters that are needed for each layer. This reduction is caused by parameter sharing.
In a traditional multi-layer perceptron weights exist for each connection while in CNNs a set of kernels are applied to the input. The kernels used in CNNs are small and are repeatedly applied across the input.
This reduces the amount of parameters the network has. 

\begin{figure}
  \includegraphics[width=8cm]{../images/cun.jpeg}
  \caption[short]{Architecture of LeNet-5: used for digit recognition and introduced by Yan LeCun \cite{lecun-1998}}
\end{figure}

\subsection{U-Net}
Semantic segmentation, the task of assigning a class label to each pixel in an image, is key to understand an image from a computer vision perspective.
This is the starting point for U-Net which was developed by Ronneberger et al. \cite{ronneberger-2015} to segment images from microscopes in biomedical applications.
U-Net is a fully convolutional architecture that provides accurate and detailed pixel-level predictions.
U-Net is specifically designed to capture both local and global context information.
The U-Net architecture gets its name from its U-shaped design (Figure \ref{fig:unet}), which consists of an encoder path and a decoder path.
The encoder path resembles a traditional CNN and serves the purpose of capturing spatial information
It is made out of multiple convolutional and pooling layers, where each convolutional layer extracts increasingly abstract features by convolving with learnable filters and applying the Rectified Linear Unit (ReLU) function \cite{relu}.
The decoder path, on the other hand, aims to recover the spatial information lost during the pooling and convolutional operations of the encoder.
It employs a series of transposed convolutional layers to gradually increase the spatial resolution.
The skip connections between the corresponding encoder and decoder layers help preserve fine-grained details that would be otherwise be lost during the encoding process.
U-Net is able to segment 2 dimensional images, however in some biomedical contexts 3D scans are used. To segment these kinds of inputs 3D U-Net was introduced by Çiçek et al. \cite{cicek-2016}.
3D U-Net is similar to 2D U-Net except instead of using 2D pooling and 2D convolution layers it uses it's 3 dimensional counterparts. Authors such as Brodehl. \cite{predictionLightning} have shown that
the depth dimension in a 3D U-Net can be replaced by the time dimension which leads to capturing time based context information.

\begin{figure}
  \includegraphics[width=8cm]{../images/unet.png}
  \caption[short]{U-Net architecture, encoder and decoder sections of the model form the U structure. Encoder Passes context information to the decoder through \textit{skip connections}. \cite{ronneberger-2015}}
  \label{fig:unet}
\end{figure}

\subsection{Convolutional LSTM}
As mentioned in the related works, the LSTM model, introduced by Hochreiter et al. \cite{lstm}.
The improvement of this recurrent neural network design over prior designs is that a long term memory path was introduced.
The long term memory path can be operated on by the input sequences, the operations that are performed on the long term memory
depend on \textit{gates}. There are two gates that affect the long term memory: the input gate which decides which new information will be added and the forget gate, which decides how much will be forgotten.
LSTM works with vectors, this would mean that to process images in a LSTM these would need to be flattened to a vector, most spatial information would be lost. This is why Shi et al. introduced ConvLSTM \cite{convlstm}.
Which replaces the learnable vectors in the LSTM with kernels, and the operations become convolutions.
The equations of the ConvLSTM cell are below (equation 1-4). Each ConvLSTM cell produces a short term memory $ \mathcal{H}_{t}$ and a long term memory $\mathcal{C}_{t}$,
the output after passing all of our sequence to the model is the short term memory $\mathcal{H}_{t}$. 
The formulas can be explained starting with the use of activation functions $\sigma$ and $tanh$, sigmoid and hyperbolic tangent.
The sigmoid function maps any $x$ between 0 and 1, values between 0 and 1 are used here as the \textit{fraction of information that is added or removed}.
On the other hand the hyperbolic tangent function maps an input $x$ between $-1$ and 1. Therefore when the sigmoid activation represents the fraction or percentage of information being added or removed from for example the
long term memory while the $tanh$ function is being used to normalize the information itself between 0 and 1.
Take for example the equation for the forget gate (equation 2), noting that $*$ notates a convolution operation.
The result of this equation is a matrix with values between 0 and 1 due to the sigmoid function.
The $f_t$ is used in equation 3, it is used to calculate the element wise multiplication notated $\odot$
with the previous long term memory. This allows only a fraction of the long term memory, to continue into the output of the
current cell and further cells in the unrolled Convolutional LSTM. This is why it is called the forget gate. 

\begin{equation}
  i_{t} = \sigma\left(W_{xi} * X_{t} + W_{hi} * H_{t-1} + W_{ci} \odot \mathcal{C}_{t-1} + b_{i}\right)\\
\end{equation}

\begin{equation}
  f_{t} = \sigma\left(W_{xf} * X_{t} + W_{hf} * H_{t-1} + W_{cf} \odot \mathcal{C}_{-1} + b_{f}\right)
\end{equation}

\begin{equation}
  \mathcal{C}_{t} = f_{t} \odot \mathcal{C}_{t-1} + i_{t} \odot \text{tanh}\left(W_{xc} * X_{t} + W_{hc} * \mathcal{H}_{t-1} + b_{c}\right)
\end{equation}

\begin{equation}
  o_{t} = \sigma\left(W_{xo} * X_{t} + W_{ho} * \mathcal{H}_{t-1} + W_{co} \odot \mathcal{C}_{t} + b_{o}\right)\\
\end{equation}

\begin{equation}
  \mathcal{H}_{t} = o_{t} \odot \text{tanh}\left(C_{t}\right)\\
\end{equation}

\subsection{Attention}
The concept of the attention mechanism in machine learning comes from Bahadanau et al. \cite{bahdanau-2015}. The authors were investigating how to assign different levels of importance to each input in a sequence to sequence model.
Usually attention is computed through a shallow multi-layer perceptron (MLP) since it is expensive to add attention to a model.
However this shallowness is not enough when it comes to computer vision models. Because we are dealing with inputs that are images the amount of connection becomes $(h\times w)^2$
due to the fact that every neuron in the input layer must be connected to the neuron in the output layer.
To address this issue, Axial Attention was introduced by Ho et al. \cite{DBLP:journals/corr/abs-1912-12180} as a means of alleviating this problem.
Axial attention simplifies the computation of attention by exclusively considering the adjacent row and column, thus mitigating the exponential growth of parameters.

\subsection{Radar Data}
Radar instruments are employed to measure precipitation levels within a given area.
These devices operate from ground level and emit microwave pulses while rotating.
As these pulses encounter atmospheric particles, they disperse in various directions.
A portion of the energy emitted by the radar is reflected back and recorded.
By employing the relationship between speed, time, and distance, the radar can determine the particle's proximity to itself.
The quantity of energy that returns to the radar following interaction with precipitation is termed \textit{reflectivity} denoted by $Z$.
To assess the rainfall rate in millimeters per hour, the Marshall Palmer Relationship, \cite{marshall-1948} is utilized to convert reflectivity factor.
Meteorologists often prefer to use decibels relative to $Z$ as a more convenient unit.
This unit expresses reflectivity relative to a 1 millimeter drop within a cubic meter of volume ($Z_0$) \cite{rogers-1976}.

\begin{equation}
  dBZ = 10 \times log_{10}(\frac{Z}{Z_0})
\end{equation}

See table \ref{tab:data} for a table with dBZ values and their corresponding meterological interpretation.
The range of a radar extends to a couple hundred kilometers from their origin which means that to observe a larger area a group of radars
can be combined to form a composite radar image see figure \ref{fig:radsource}. 


\subsection{Satellite Data}
Satellites play a crucial role in meteorology as they provide valuable observations of the Earth from above.
Geostationary satellites orbit the earth at the same speed as the planet's rotation on it's own axis, which allow them to be fixed relative to a given longitude.
This class of satellites orbit the earth at an altitude of $\approx 35786 km$.
Worldwide coverage can be achieved by deploying multiple satellites, for example EUMETSAT operates MSG satellites for Europe, NOAA operates GOES satellites for the Americas and the Himawari satellite is operated by the Japan Meteorological Agency, which covers the Asia-Pacific region.
Focusing on Meteosat-10 satellite from which we obtain our data, this satellite produces a scan every 15 minutes.
We obtain 11 satellite images corresponding to different spectral channels (see figure \ref{fig:satchannels}). Three main types of channels are available: visible, infrared and water vapour channels (See table \ref{tab:channels}). 
Visible channels are measured by the satellite when radiation from the sun reflects on the earth's surface, infrared channels on the other hand receive radiation emitted by the earth and clouds allowing for imagery even at nighttime. The wave vapour channels on the other hand, capture radiation emitted
by water vapour in the upper troposphere.


\section{Methodology}

In this section, we will provide explanations of the methods, techniques, and procedures employed in our experiments. We will begin by discussing the data preprocessing steps, followed by an explanation of the training process. Finally, we will describe how we evaluated the performance of our models.

\subsection{Engineering For Machine Learning}
In order to streamline our experimental process and minimize effort,
we focused on developing the training and preprocessing code in a way that allows for easy experimentation.
We also utilized tools that enhance efficiency and reproducibility in the research process, reducing time-consuming tasks.
\medskip

We adopted the \texttt{zenml} \cite{zenml} framework to structure our training and preprocessing pipelines.
This framework brings several advantages to our workflow.
One notable benefit is the promotion of modularity in the design of our pipelines.
Each pipeline consists of individual steps, which enhances the code's modularity.
By defining a series of functions with clear inputs and outputs,
we ensure that each step can be easily understood and modified as needed.
Moreover, the \texttt{zenml} web application offers a convenient way to monitor the status of our pipelines.
This feature provides transparency and improves comprehension by providing insights into the outputs generated at each step.
\medskip

For experiment tracking we used the \texttt{MLFlow} framework \cite{mlflow}.
This makes it easy to track all metrics over all experiments.
With \texttt{MLFlow} it is also possible to compare different parameters that were used during training to experimentally find the best combinations.
In \texttt{MLFlow} we save each finished model as an artifact which can be directly served as a server endpoint to start providing end users access to our models.
\medskip

We used the pytorch lightning library. Using patterns such as the \texttt{DataModule} and {LightningModule} to accelerate the
research process. This library abstracts the process of backpropagation and updating parameters.
Furthermore pytorch ligthning \cite{lightning} handles
switching from training devices automatically for example \texttt{cpu} and \texttt{cuda} devices
which decouples the training code from specific hardware.

We made use of the \texttt{typed-settings} library \cite{typed} to allow cleanly structuring and validating settings for models.
This ensures that to train a new version of a model in most cases only adjustments to the configuration file needs to be done.
\texttt{typed-settings} supports passing training settings through toml configuration files, environment variables and command line options.

\subsection{Data Preprocessing}
For the preprocessing of data we created a pipeline which distinguishes between satellite images and radar images to process each following their own needs (See figure \ref{fig:preprocessing}).
\medskip

The pipeline begins by obtaining all necessary files, from a remote storage bucket.
This is done by a Bucket Service class which uses the \texttt{boto3} \cite{boto}
library to interface with the bucket and download files in the required date ranges.
\medskip

In the case of satellite data, the obtained files are compressed in zip files. The pipeline handles the extraction of these files deleting any files which are not needed along the way to ease the storage requirements.
Then we \textit{reproject} the satellite images using the \texttt{satpy} package. This downsampling is done by a combination of cropping and interpolation via the nearest-neighbor algorithm.
\medskip

By reprojecting we reduce the dimensions of each satellite image to \texttt{256 x 256} pixels from it's original dimensions of \texttt{3712 x 3712}.
We also obtain only the geographical area of interest, specified by the coordinates for the lower corner \texttt{(50°0'0"N 0°0'0"E)} and the upper corner \texttt{(55°0'0"N 10°0'0"E)} of the region, this gives us
the area centered on the netherlands with other bordering countries (see figure \ref{fig:reprojection}). Additionally we change the map projection to \textit{Mercator} from the original \textit{Geostationary}.
This is done to have both input and target grids in the same map projections.
After reprojecting we perform a \textit{statistics} step where we aggregate the dataset by finding the minimum and maximum values for each channel see table \ref{tab:channels} for the list of all channels.
The statistics are necessary for the next step which is normalization. During the normalization step we perform the \textit{Min-Max} Normalization (equation 1).

\begin{figure}
  \centering
  \includegraphics[width=225pt]{./images/reprored.png}
  \caption{Reprojection of Satellite Data: Converting from geostationary projection to mercator projection and reduce to area of interest.}
  \Description{Satellite Image of the earth}
  \label{fig:reprojection}
\end{figure}


\begin{equation}
  x_{normalized} = \frac{x-min(\bar{x})}{max(\bar{x})-min(\bar{x})}
\end{equation}

After finalizing the normalization step we sample an image from the dataset which is visualized to check for errors in the pipeline.
\medskip

The radar pipeline begins with the downloaded \texttt{h5} files
each is converted to decibels relative to Z (dBZ) (equation 8) from a \textit{grayscale} unit ranging from 0 to 255.
Using decibels relative to Z adds to the interpretability  and makes it easier to weight loss functions and metrics
according to the level of precipitation. The drawback of this is that to plot the image or perform image transformations many
existing libraries make the assumption of either \textit{grayscale} or \textit{rgb} images, therefore to make use of these resources we must sometimes convert back to the \textit{grayscale} unit. 

\begin{equation}
  dBZ(x) = x \cdot 0.5 - 32
\end{equation}

The preprocessing pipeline then splits into two, one step will normalize values between 0 and 1 using \textit{Min-Max} normalization and the other step will use levels of \textit{dBZ} to create discrete ranges of precipitation to use as classes during training \ref{tab:classes}.
Finally the current images in the pipeline are resized using the nearest neighbor algorithm, to avoid changing the class values with bilinear interpolation.
Next identically to the satellite pipeline we sample and visualize a radar image for verification purposes.

\begin{figure}
  \centering
  \includegraphics[width=225pt]{./images/bins.png}
  \caption{Preprocessed Radar Reflectivity Data For 2023-03-10 at 11:55 UTC. Left with normalized data and right with pixels put into 8 different rainfall intensity classes.}
  \Description{Satellite Image of the earth}
  \label{fig:bins}
\end{figure}


\begin{figure}
  \centering
  \includegraphics[width=225pt]{./images/prepro.png}
  \caption{Data Preprocessing Pipeline: Satellite Data and Radar data preprocessed separately}
  \Description{Satellite Image of the earth}
  \label{fig:preprocessing}
\end{figure}

\subsection{Proposed Model Architectures}

The 3D U-Net based model is given a input patch of dimension $8\times11\times256\times256$. The U-Net model produces a segmented image for each slice of the depth dimension.
Therefore we added an additional 3D convolution after the normal U-Net architecture which reduces the output of the network from the shape $8\times8\times256\times256$ to the desired $1\times8\times256\times256$ or in the case of regression to $1\times256\times256$
\medskip


For ConvLSTM models, we start with 3 layers of Convolutional LSTMs
with 64 filters that use a kernel size of 3.
Then the short term memory of the last layer is passed to three 2D Convolutional Layers.
The first Convolutional layer reduces the amount of channels to 32, the second to 16 and the last one to either 8 or 1 for classification and regression respectively.
In the intermediate layers a ReLU \cite{relu} activation function is performed.

\subsection{Experimental Setup}
We trained each model for 50 epochs. The batch size was set to 1 because of memory constraints. We used the \textsc{Adam} algorithm for optimization from Kingma et al \cite{kingma-2014}.
All model were trained on a single NVIDIA GeForce RTX 2070 SUPER GPU. We used different losses for classification and regression.
First we used Multi-Class CrossEntropyLoss for classification this first applies a log softmax activation function which normalizes the
outputs produced by our models. Then it calculates the cross entropy loss between the normalized input and the target.
For regression we used the MSE loss. To address class imbalance in the data we added weights for the CrossEntropyLoss. These weights were
obtained by finding the frequencies of the classes in the dataset as follows:

\begin{equation}
  weight(c) = 1 - \frac{1}{h\times w \times n} \times \sum_{i=1}^n \sum_{j=1}^h \sum_{k=1}^w [Y_{ijk} = c]
\end{equation}

We decided to conduct our experiments using $t = 5$, allowing for 1 hour and 15 minutes of temporal data, $h = 256$, $w = 256$, and $c = 12$.

We created 2 different types of datasets for use in the experiments.
The first type is a sequence dataset, this dataset does not repeat any satellite files. It increments the starting point at each sample
by the length of the satellite sequence. On the other hand to make use of all the available data we created a sliding sequence dataset which increments the starting point of the sequences by 1 meaning that the sequence moves by
1 satellite image forward. We have also handled aligning the satellite and corresponding radar target based on their timestamps. 

\subsection{Performance Evaluation}

To evaluate the performance of our models we implemented several metrics. Each of these metrics can be used to measure
the quality of a trained model. Utilizing a range of metrics allows us to better analyze the performance of a particular machine learning model.
Metrics that are used to measure the performance of classification problems are different from the metrics that are used for regression. Note that we
give the formulas that are used to calculate the metric for each image, in the reported results these metrics have been averaged over the entire test dataset.

\subsubsection{Regression Evaluation Metrics}
Regression metrics are often error functions, these functions calculate the distance between prediction and target values. 
When predicting more than one value, we can extend the definition of an error function
by calculating the mean or sum of all distances, the distances themselves are calculated with a certain distance metric for 2 values that are in the same position in the prediction and target.
This is an important distinction when considering that we predict a $h \times w$ image, to calculate the distance between the prediction and target, we will calculate the distance between each pixel then divide it by the number of pixels in the image.

The simplest metric is the \textit{Mean Absolute Error} (equation 3). It takes the absolute value
of the distance between to values since the sign of the error does not affect it's importance.
The mean average error is not differentiable at $x = 0$.

\begin{equation}
  MAE = \frac{1}{h \times w}\sum_{i=0}^h\sum_{j=0}^w |\hat{y_{ij}} -y_{ij}|
\end{equation}

A common regression metric is \textit{Mean Squared Error} (equation 4). MSE squares the distance between two values, so it is always positive and
becomes larger exponentially faster than MAE as the distance between the two values increases.

\begin{equation}
  MSE = \frac{1}{h\times w}\sum_{i=0}^h\sum_{j=0}^w (\hat{y_{ij}} -y_{ij})^2
\end{equation}

MSE punishes outliers more severely and is harder to interpret than MAE because the unit of the error is the squared original unit for example in our case (dBZ) becomes ($dBZ^2$).
The root mean squared error solves the problem of MSE producing uninterpretable units by taking the root of the MSE (equation 5).
\begin{equation}
RMSE = \sqrt{MSE}
\end{equation}

Some metrics have been designed for the task of precipitation nowcasting itself, what is particular about this case is that we place more
importance on predicting the outliers than the overall data. This is because extreme values of rain are the most important to predict correctly as these cause the
most damage to society. Shi et al. \cite{shi2017deep} created \textit{BMAE} (equation 6) and \textit{BMSE} (equation 7) because of this.
These metrics weight errors higher as the target pixel value increases. 

\begin{equation}
BMAE = \frac{1}{h\times w}\sum_{i=0}^h\sum_{j=0}^w weight(y_{ij}) \cdot |\hat{y_{ij}} - y_{ij}|
\end{equation}

\begin{equation}
BMSE = \frac{1}{h\times w}\sum_{i=0}^h\sum_{j=0}^w weight(y_{ij}) \cdot (\hat{y_{ij}} -y_{ij})^2
\end{equation}


\subsubsection{Classification Evaluation metrics} Many classification metrics are based on the \textit{Confusion Matrix}.
The simplest case is a binary classification here a model predicts between two classes suppose we call them 
\textit{positive} and \textit{negative}. There are two possibilities for this output
it can be either \textit{true} or \textit{false}. Thus there exist 4 disjoint sets ($TP$, $TN$, $FP$ and $FN$) all subsets of $\hat{Y}$ the set of all predictions.
Calculating values based on the cardinality of these sets helps us understand the performance of a classification model.
Accuracy is a intuitive metric, it is defined as the fraction of correct classifications over the total amount of classifications.  

\begin{equation}
  Accuracy = \frac{|TP| + |TN|}{|TP| + |FP| + |TN| + |FN|}
\end{equation}


The precision brings the focus on predicting a false positive.
Precision can be a useful measure when we try to measure if a model
is giving too many incorrect positive predictions.

\begin{equation}
  Precision = \frac{|TP|}{|TP| + |FP|}
\end{equation}

The recall gives the score focusing on when the model predicts a false negative.
Recall can be useful when we want to know if the model is predicting too many incorrect negative predictions.
\begin{equation}
  Recall = \frac{|TP|}{|TP| + |FN|}
\end{equation}

The F1 score captures the trade-off between precision and recall, offering a single metric to measure the model's effectiveness.

\begin{equation}
  F1Score = 2 \times \frac{Precision \times Recall}{Precision + Recall}
\end{equation}

The Jaccard index, also known as intersection over union (IoU) created by Jaccard \cite{paul},
Can be used to evaluate a model's performance particularly for image segmentation,
it quantifies the overlap between the predicted positive instances and the actual positive instances.
The Jaccard index is particularly useful when evaluating models on imbalanced datasets, where it can provide a robust measure of performance by focusing on the correct prediction of positive instances while discounting true negatives.

\begin{equation}
  Jaccard Index = \frac{|TP|}{|TP| + |FP| + |FN|}
\end{equation}
\medskip

These metrics can be extended from the binary classification case that has two classes to a general definition for n classes by redefining $TP$, $FP$, $FN$ and $TN$.

Another important remark is that for images, since each pixel is classified, the metrics are first calculated at the image level, then averaged over the number of samples.

\section{Results}
% The metrics for our experiments are available in table 1. The scores are very high for Accuracy, Recall, F1Score and Precision (>= 90), we even 
% have for ConvLSTM 53\% exact matches, which means that the predicted image was exactly the ground truth that occurred.
% however this does not mean that the model is correct, due to a very high data imbalance towards 0 or no rain in the dataset.
% The most important metric that actually quantifies the performance of the model is the Jaccard Index. The best performing model following this metric is the plain ConvLSTM model
% with a Convolutional head. If we compare both U-Net and ConvLSTM models visually \ref{fig:convclass} \ref{unet} we can see that ConvLSTM is more confident in predicting high rain intensity
% while the U-Net model predicts only low levels of rain, however it is more accurate at predicting the area of rain in the ground truth. Another remark is that all metrics seem to be the same across the models.
% This can occur if the false positive rate is equal to the false negative rate.
In this section we will present the results of the experiments.

\subsection{Data}
After preprocessing we obtain \texttt{3331} satellite files and \texttt{10348} radar files see table (\ref{tab:data}).
The storage size of which is equal to \texttt{387.6} Gigabytes.
We split the data with a \texttt{0.8} split for testing \texttt{0.1} for validation and testing.
A sample of a preprocessed satellite file can be seen in Figure \ref{fig:satchannels}.
From this satellite image we can see that the area has been cropped from the original.
%This image is also in a different map projection from the original, the preprocessing has produced a mercator map projection,
%from the original geostationary projection.
The area visible corresponds to the European Netherlands in the center, on the left we have south east England and on the right we have Germany and Denmark.
On the south of the image we can observe clouds in all visual and infrared channels.
On the water vapour channels we can also see that there is higher degree of water vapour in the south and middle of the image, compared to the other parts of the image.
A preprocessed radar image can be seen in Figure \ref{fig:radar-pre}. A binned radar image can be seen in Figure \ref{fig:radar-bin} where each pixel is represented by one of the eight classes (table \ref{tab:dbzz}).

\subsection{Trained Classification Models}
Trained classification models are listed in table \ref{tab:class}.
We focus on the jaccard index to evaluate the best performing model due to class imbalance.
Based on the jaccard index of \textbf{0.1249} the best performing model is the ConvLSTM without using the attention mechanism. This same model also managed to
predict 53\% of the radar images in the test set exactly. The model most likely obtains a perfect score when there is no rain in the image and the expected output therefore is a empty image.
In second place we have the ConvLSTM with attention, and in the last place we have the 3D U-Net. We have also visually analyzed the predictions made by the models which can be seen in figure \ref{fig:convclass} and figure \ref{fig:unet-pred}.
The U-Net based model predicts low amounts of rain but generally predicts the position and movement of clouds better. Meanwhile ConvLSTM based models are more eager
to predict higher rainfall classes and therefore look similar to the target, however when investigating the outputs for the entire test set it seems that the ConvLSTM based model does not learn the spatio-temporal patterns
as well as the U-Net. This can be observed in Figure \ref{fig:experiment-19}, where the LSTM based model does not segment areas covered by clouds, but seemingly randomly predicting rainfall over the whole Netherlands.
Compare this with figure \ref{fig:experiment-160} where we can clearly see that U-Net based model knows that the incoming cloud from the south west, will produce precipitation.


\subsection{Trained Regression Models}
Trained classification models are listed in table \ref{tab:reg}. Similar to the classification models the ConvLSTM
has the lowest error for the regression experiments. From an analysis of the metrics it can be seen that the U-Net based model performs worse compared to the ConvLSTM variant.
The balanced MSE and balanced MAE metrics which weight pixels with higher amounts of rain we can see that the ratio of MSE to BMSE
is $26.08$ for U-Net and only $6$ for ConvLSTM. Interestingly U-Net suffers from a bias 
in predicting low amounts of precipitation in both classification and regression.



\begin{table*}[h]
  \caption[short]{Metrics on Test Set for variants of classification models trained on 50 epochs.}
  \begin{tabular}{@{}lllllll@{}}
  \toprule
  Models               & Accuracy & Precision & Recall & F1Score & Exact Match & Jaccard Index \\ \midrule
  3D U-Net             & 0.9199   & 0.9199    & 0.9199 & 0.9199  & 0.0000      & 0.1150        \\
  \textbf{ConvLSTM}    & \textbf{0.9999}   &  \textbf{0.9999}    & \textbf{0.9999} & \textbf{0.9999}  & \textbf{0.5385}      & \textbf{0.1249}        \\
  ConvLSTM + Attention & 0.9300   & 0.9300    & 0.9300 & 0.9300  & 0.0000      & 0.1160 
  \end{tabular}
  \label{tab:class}
\end{table*}

\begin{table*}[h]
  \caption[short]{Metrics on Test Set for variants of regression models trained on 50 epochs.}
  \begin{tabular}{@{}lllllll@{}}
  \toprule
  Models                & MAE & MSE & RMSE & BMAE & BMSE \\ \midrule
  3D U-Net              & 0.366  & 	0.351  & 0.565 & 	5.148  & 9.155 \\
    \textbf{ConvLSTM}   & \textbf{0.032}   &  \textbf{0.001} & \textbf{0.034} & \textbf{0.055}  & \textbf{0.006}  \\
  ConvLSTM + Attention  & 0.181   & 0.059 & 0.242 & 0.265  & 0.159  \\
  \end{tabular}
  \label{tab:reg}
\end{table*}


\begin{figure}
    \centering
    \includegraphics[width=225pt]{./images/infrared.png}
    \caption{Satellite Image: Infrared Channel 18UTC $12.0\mu m$}
    \Description{Satellite Image of the earth}
    \label{fig:infra}
\end{figure}


\section{Conclusions}
In this paper we investigate possible methods to create a model capable of
predicting precipitation based on satellite images.
We propose and train 3 different model architectures. The trained models are benchmarked and compared based on several metrics.
All models are framed in two different tasks: classification and regression.
In the classification task we train a model to predict from 8 classes mapping to rain intensity categories. In pixel level regression we do not modify the 
radar reflectivity data and we train a pixel-level regression model.
We find that all trained models are capable of predicting some precipitation but they require improvements to perform well in practice.
This can be attributed to relatively low amounts of data being used, difference in resolution
of the two types of data, data imbalances and the complexity of the problem space. 
The model that achieves the best score over the test dataset is the ConvLSTM model. The attention mechanism
as used currently seems to make predictions worse for both classification and regression. More testing needs to be done to state certainly if ConvLSTM surpasses
the performance of 3D U-Net since form a visual standpoint U-Net appears to give more meteorologically viable results.

% \begin{acks}
% I would like to thank my supervisor from Elena Mocanu, for her help. Furthermore I would like to thank I would like to thank Dina Lazorkeno for proofreading my thesis.
% \end{acks}

\bibliographystyle{ACM-Reference-Format}
\bibliography{ref}

% \newpage
\appendix

\section{Images}

\begin{figure}[hbp]
  \centering
  \includegraphics[width=225pt]{./images/radar_reflectivity.png}
  \caption{Radar Reflectivity}
  \Description{}
  \label{fig:reflect}
\end{figure}

\begin{figure}[hbp]
  \centering
  \includegraphics[width=225pt]{./images/vis_006.png}
  \caption{Satellite Image: Visible Channel 18UTC $0.6\mu m$}
  \Description{}
  \label{fig:vis}
\end{figure}

\begin{figure}[hbp]
  \centering
  \includegraphics[width=225pt]{./images/radar_source.png}
  \caption{Satellite Image: Infrared Channel 18UTC $12.0\mu m$}
  \Description{}
  \label{fig:source}
\end{figure}

\section{Tables}

\begin{table}[hbp]
  \caption{Available Satellite Channels.}
  \begin{tabular}{@{}lll@{}}
  \toprule
  Channel & Type        & $\lambda$ \\ \midrule
  VIS006  & Visual      & 0.6 mm      \\
  VIS008  & Visual      & 0.8 mm      \\
  IR\_016 & Infrared    & 1.6 mm      \\
  IR\_039 & Infrared    & 3.9 mm      \\
  IR\_087 & Infrared    & 8.7 mm      \\
  IR\_097 & Infrared    & 9.7 mm      \\
  IR\_108 & Infrared    & 10.8 mm     \\
  IR\_120 & Infrared    & 12.0 mm     \\
  IR\_134 & Infrared    & 13.4 mm     \\
  WV\_062 & Water Vapor & 6.2 mm      \\
  WV\_073 & Water Vapor & 7.3 mm      \\ \bottomrule
  \end{tabular}
  \label{tab:channels}
\end{table}

\begin{table}[h]
\caption{Reflectivity in dBZ versus Rainrate}
\begin{tabular}{@{}llll@{}}
\toprule
LZ(dBZ) & R(mm/h) & R(in/h)        & Intensity             \\ \midrule
5       & (mm/h)  & \textless 0.01 & Hardly noticeable     \\
10      & 0.15    & \textless 0.01 & Light mist            \\
15      & 0.3     & 0.01           & Mist                  \\
20      & 0.6     & 0.02           & Very light            \\
25      & 1.3     & 0.05           & Light                 \\
30      & 2.7     & 0.10           & Light to moderate     \\
35      & 5.6     & 0.22           & Moderate rain         \\
40      & 11.53   & 0.45           & Moderate rain         \\
45      & 23.7    & 0.92           & Moderate to heavy     \\
50      & 48.6    & 1.90           & Heavy                 \\
55      & 100     & 4              & Very heavy/small hail \\
60      & 205     & 8              & Extreme/moderate hail \\
65      & 421     & 16.6           & Extreme/large hail    \\ \bottomrule
\end{tabular}
\end{table}


\section{Listings}
\begin{listing}
  \begin{minted}{toml}
    [model]
    name = "SAT2RAD_UNET"
    classes=8
    
    [model.input_size]
    height = 256
    width = 256
    channels = 12
    sequence_length = 8
    
    [model.output_size]
    height = 256
    width = 256
    channels = 1
    sequence_length = 1
    
    [model.unet]
    kernel_size = [3, 3]
    layers = 3
    filters = 64
    
    [model.training]
    max_epochs = 100
    class_weights = [
                0.01081153,
                0.13732371,
                0.13895907,
                0.1416087,
                0.14272867,
                0.14285409,
                0.14285709,
                0.14285714,
            ]
    metrics = [
        'acc',
        'precision',
        'recall',
        'exact',
        'f1', 
        'jaccard'
    ]
    
    [mlflow]
    experiment_name = "sat2rad_unet"
    experiment_tracker = "Infoplaza MLFlow"
    
    
    [visualize]
    output_dir = '../../../../../logs/'
  \end{minted}
  \caption{toml configuration file for U-Net Model.}
  \label{lst:hello}
\end{listing}

\end{document}
