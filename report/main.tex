 %% IMPORTANT NOTICE:
%% 
%% For the copyright see the source file.
%% 
%% Any modified versions of this file must be renamed
%% with new filenames distinct from sample-manuscript.tex.
%% 
%% For distribution of the original source see the terms
%% for copying and modification in the file samples.dtx.
%% 
%% This generated file may be distributed as long as the
%% original source files, as listed above, are part of the
%% same distribution. (The sources need not necessarily be
%% in the same archive or directory.)
%%
%% Commands for TeXCount
%TC:macro \cite [option:text,text]
%TC:macro \citep [option:text,text]
%TC:macro \citet [option:text,text]
%TC:envir table 0 1
%TC:envir table* 0 1
%TC:envir tabular [ignore] word
%TC:envir displaymath 0 word
%TC:envir math 0 word
%TC:envir comment 0 0
%%
%%
%% The first command in your LaTeX source must be the \documentclass command.
%%%% Small single column format, used for CIE, CSUR, DTRAP, JACM, JDIQ, JEA, JERIC, JETC, PACMCGIT, TAAS, TACCESS, TACO, TALG, TALLIP (formerly TALIP), TCPS, TDSCI, TEAC, TECS, TELO, THRI, TIIS, TIOT, TISSEC, TIST, TKDD, TMIS, TOCE, TOCHI, TOCL, TOCS, TOCT, TODAES, TODS, TOIS, TOIT, TOMACS, TOMM (formerly TOMCCAP), TOMPECS, TOMS, TOPC, TOPLAS, TOPS, TOS, TOSEM, TOSN, TQC, TRETS, TSAS, TSC, TSLP, TWEB.
% \documentclass[acmsmall]{acmart}

%%%% Large single column format, used for IMWUT, JOCCH, PACMPL, POMACS, TAP, PACMHCI
% \documentclass[acmlarge,screen]{acmart}

%%%% Large double column format, used for TOG
\documentclass[acmtog, authorversion]{acmart}
\usepackage[htt]{hyphenat}
\usepackage{graphicx}
\usepackage{lipsum}
\usepackage{pgfgantt}
\graphicspath{{images/}}

%%%% Generic manuscript mode, required for submission
%%%% and peer review
%\documentclass[manuscript,screen,review]{acmart}
%% Fonts used in the template cannot be substituted; margin 
%% adjustments are not allowed.
%%
%% \BibTeX command to typeset BibTeX logo in the docs
\AtBeginDocument{%
  \providecommand\BibTeX{{%
    \normalfont B\kern-0.5em{\scshape i\kern-0.25em b}\kern-0.8em\TeX}}}

%% Rights management information.  This information is sent to you
%% when you complete the rights form.  These commands have SAMPLE
%% values in them; it is your responsibility as an author to replace
%% the commands and values with those provided to you when you
%% complete the rights form.

\setcopyright{acmcopyright}
\copyrightyear{2023}
\acmYear{2023}

%% These commands are for a PROCEEDINGS abstract or paper.
\acmConference[TScIT 39]{39$^{th}$ Twente Student Conference on IT}{July 8,
  2023}{Enschede, The Netherlands}
%
%  Uncomment \acmBooktitle if th title of the proceedings is different
%  from ``Proceedings of ...''!
%
%\acmBooktitle{Woodstock '18: ACM Symposium on Neural Gaze Detection,
% June 03--05, 2018, Woodstock, NY} 
%\acmPrice{15.00}
%\acmISBN{978-1-4503-XXXX-X/18/06}


%%
%% Submission ID.
%% Use this when submitting an article to a sponsored event. You'll
%% receive a unique submission ID from the organizers
%% of the event, and this ID should be used as the parameter to this command.
%%\acmSubmissionID{123-A56-BU3}

%%
%% For managing citations, it is recommended to use bibliography
%% files in BibTeX format.
%%
%% You can then either use BibTeX with the ACM-Reference-Format style,
%% or BibLaTeX with the acmnumeric or acmauthoryear sytles, that include
%% support for advanced citation of software artefact from the
%% biblatex-software package, also separately available on CTAN.
%%
%% Look at the sample-*-biblatex.tex files for templates showcasing
%% the biblatex styles.
%%

%%
%% The majority of ACM publications use numbered citations and
%% references.  The command \citestyle{authoryear} switches to the
%% "author year" style.
%%
%% If you are preparing content for an event
%% sponsored by ACM SIGGRAPH, you must use the "author year" style of
%% citations and references.
%% Uncommenting
%% the next command will enable that style.
%%\citestyle{acmauthoryear}

%%
%% end of the preamble, start of the body of the document source.
\begin{document}

%%
%% The "title" command has an optional parameter,
%% allowing the author to define a "short title" to be used in page headers.
\title{Satellite to Radar: Sequence to Sequence Learning for precipitation forecasting}

%%
%% The "author" command and its associated commands are used to define
%% the authors and their affiliations.
%% Of note is the shared affiliation of the first two authors, and the
%% "authornote" and "authornotemark" commands
%% used to denote shared contribution to the research.

\author{Mark Bruderer}
\email{m.a.bruderervanblerk@student.utwente.nl}
\affiliation{%
  \institution{University of Twente}
  \streetaddress{P.O. Box 217}
  \city{Enschede}
  \country{The Netherlands}
  \postcode{7500AE}
}



%%
%% By default, the full list of authors will be used in the page
%% headers. Often, this list is too long, and will overlap
%% other information printed in the page headers. This command allows
%% the author to define a more concise list
%% of authors' names for this purpose.
\renewcommand{\shortauthors}{Mark Bruderer}

%%
%% The abstract is a short summary of the work to be presented in the
%% article.
\begin{abstract}
  \lipsum[2-3]
\end{abstract}

%%
%% The code below is generated by the tool at http://dl.acm.org/ccs.cfm
%% Please copy and paste the code instead of the example below.
%%Optional
%%
%%\begin{CCSXML}
%%<ccs2012>
%% <concept>
%%  <concept_id>10010520.10010553.10010562</concept_id>
%%  <concept_desc>Computer systems organization~Embedded systems</concept_desc>
%%  <concept_significance>500</concept_significance>
%% </concept>
%% <concept>
%%  <concept_id>10010520.10010575.10010755</concept_id>
%%  <concept_desc>Computer systems organization~Redundancy</concept_desc>
%%  <concept_significance>300</concept_significance>
%% </concept>
%% <concept>
%%  <concept_id>10010520.10010553.10010554</concept_id>
%%  <concept_desc>Computer systems organization~Robotics</concept_desc>
%%  <concept_significance>100</concept_significance>
%% </concept>
%% <concept>
%%  <concept_id>10003033.10003083.10003095</concept_id>
%%  <concept_desc>Networks~Network reliability</concept_desc>
%%  <concept_significance>100</concept_significance>
%% </concept>
%%</ccs2012>
%%\end{CCSXML}

%%\ccsdesc[500]{Computer systems organization~Embedded systems}
%%\ccsdesc[300]{Computer systems organization~Redundancy}
%%\ccsdesc{Computer systems organization~Robotics}
%%\ccsdesc[100]{Networks~Network reliability}

%%
%% Keywords. The author(s) should pick words that accurately describe
%% the work being presented. Separate the keywords with commas.
\keywords{Machine Learning, Sequence to Sequence, Radar, Sattelite, Storms, Forecasting}

\settopmatter{printacmref=false}

%% A "teaser" image appears between the author and affiliation
%% information and the body of the document, and typically spans the
%% page.
\begin{teaserfigure}
  \includegraphics*[width=\textwidth, trim= 0in 16.0in 0in 16.0in]{./images/storm.jpg}
  \caption{Plane passing through thunderstorm. - Tom Barrett}
  \Description{Enjoying the baseball game from the third-base
  seats. Ichiro Suzuki preparing to bat.}
  \label{fig:teaser}
\end{teaserfigure}

%%
%% This command processes the author and affiliation and title
%% information and builds the first part of the formatted document.
\maketitle

\section{Introduction}

\section{Main Contribution}
provide a way of predicting radar composites from satellite images. Investigate architectures that allow the best way to do this.

\section{Related Works}

\section{Background}

\section{Methods}

\section{Expected Results}


\section{Time Planning}

\medskip

\begin{ganttchart}[hgrid,vgrid,]{1}{12}
  \gantttitle{\textbf{Planning Table}}{12} \\
  \gantttitlelist{1,...,12}{1} \\
  \ganttgroup{Crispdm}{1}{3} \\
  \ganttbar{Proposal}{1}{2} \\
  \ganttlinkedbar{Task 2}{3}{8} \ganttnewline
  \ganttbar{Final Task}{8}{12}
  \ganttlink{elem2}{elem3}
\end{ganttchart}



\section{Conclusions}




%%
%% The acknowledgments section is defined using the "acks" environment
%% (and NOT an unnumbered section). This ensures the proper
%% identification of the section in the article metadata, and the
%% consistent spelling of the heading.
\begin{acks}
This section is optional; it is a location for you to acknowledge
grants, funding, editing assistance and what have you.  In the
present case, for example, the authors would like to thank Gerald
Murray of ACM for his help in codifying this \textit{Author's
Guide} and the \textbf{.cls} and \textbf{.tex} files that it
describes.
\end{acks}

%%
%% The next two lines define the bibliography style to be used, and
%% the bibliography file.
\bibliographystyle{ACM-Reference-Format}
\bibliography{sample-base}

%%
%% If your work has an appendix, this is the place to put it.
\appendix
%Appendix A
\section{Headings in Appendices}
The rules about hierarchical headings discussed above for the body
of the article are different in the appendices. In the
\textbf{appendix} environment, the command \textbf{section} is
used to indicate the start of each Appendix, with alphabetic order
designation (i.e. the first is A, the second B, etc.) and a title
(if you include one).  So, if you need hierarchical structure
\textit{within} an Appendix, start with \textbf{subsection} as the
highest level. Here is an outline of the body of this document in
Appendix-appropriate form:
\subsection{Appendix A.1}
Sample text.
\subsection{Appendix A.2}
Sample text.
\subsubsection{Appendix A.2.1}
Sample text.

\paragraph{Inline (In-text) Equations}
\
% This next section command marks the start of
% Appendix B, and does not continue the present hierarchy
\section{Appendix B}
Sample text.
%\balancecolumns % GM June 2007
% That's all folks!


\end{document}
\endinput
%%
%% End of file `sample-authordraft.tex'.
