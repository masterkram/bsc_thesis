\documentclass[acmtog, authorversion]{acmart}
\usepackage[htt]{hyphenat}
\usepackage{graphicx}
\usepackage{lipsum}
\usepackage{pgfgantt}
\usepackage{enumitem}

\newlist{questions}{enumerate}{2}
\setlist[questions,1]{label=\textbf{RQ\arabic*.},ref=RQ\arabic*}
\setlist[questions,2]{label=(\alph*),ref=\thequestionsi(\alph*)}
\graphicspath{{images/}}

\AtBeginDocument{%
  \providecommand\BibTeX{{%
    \normalfont B\kern-0.5em{\scshape i\kern-0.25em b}\kern-0.8em\TeX}}}

\setcopyright{acmcopyright}
\copyrightyear{2023}
\acmYear{2023}

\acmConference[TScIT 39]{39$^{th}$ Twente Student Conference on IT}{July 8,
  2023}{Enschede, The Netherlands}

\begin{document}

\title{Satellite to Radar: Sequence to Sequence Learning for precipitation nowcasting}

\author{Mark Bruderer}
\email{m.a.bruderervanblerk@student.utwente.nl}
\affiliation{%
  \institution{University of Twente}
  \streetaddress{P.O. Box 217}
  \city{Enschede}
  \country{The Netherlands}
  \postcode{7500AE}
}

\renewcommand{\shortauthors}{Mark Bruderer}

\begin{abstract}
\section*{Abstract}
In this study we present SatToRad a vision transformer network for satellite based radar prediction.
\end{abstract}

\keywords{Machine Learning, Sequence to Sequence, Radar, Satellite, Storms, Forecasting}

\settopmatter{printacmref=false}

\begin{teaserfigure}
  \includegraphics*[width=\textwidth, trim= 0in 16.0in 0in 16.0in]{./images/storm.jpg}
  \caption{Plane flying through thunderstorm.^1}
  \Description{Plane flying through thunderstorm}
  \label{fig:teaser}
\end{teaserfigure}

\maketitle

\section{Introduction}
% explain what weather forecasts are:
Weather forecasts are essential for everyday operations of several sectors of society. These forecasts generate valuable outputs that can be leveraged for proactive planning and preventive measures, including automated protection for greenhouses, support for event management and safeguarding of maritime operations.

% explain what storms are:
% - why are storms important
% - what are storms => how they from
A particularly strong threat is posed by rain storms and thunderstorms. Storms are one of the most destructive weather events in nature, capable of destroying human structures and even lead to loss of life \cite{noaa-national-severe-storms-laboratory-no-date}, as such they one of the most important phenomena to accurately predict with the largest forecast time horizon possible.

% explain how to currently predict storms
At present meteorologists hold a range of tools to predict storms. Satellite imagery can be used to spot rapid cloud formation or to read the temperature at the top of the cloud giving a possible indication of the cloud's altitude. Meteorologists also make use of radar reflectivity to measure the amount of precipitation in an area. Finally meteorologists may use Numerical Weather Prediction (\textsc{NWP}) models to predict storms. \textsc{NWP} models are programs that take as input data from different layers of the atmosphere, and predict the state of the atmosphere at a future time. These programs function by simulating the atmosphere based on physics and the available readings of meteorological variables. State of the art \textsc{NWP} models are computationally expensive to run, especially for short term forecasts known as nowcasts.

% explain machine learning approaches
Machine Learning \textsc{ML} approaches have also been developed to predict extreme weather events, and have even surpassed the accuracy of Numerical Weather Predictions for this application. Machine Learning models are also faster to run than \textsc{NWP} models which can take hours to days to run, thus ML models are more suitable for real-time or near-real-time predictions, such as disaster response and energy management.
% Recently many different data driven approaches have been developed. These approaches are based on machine learning, and are trained as self-supervised models that learn patterns in the 

% explain our approach
An alternative to these is for nowcasting to have a machine learning model learn the patterns used by meteorologists to predict storms while using satellite data. i.e the rapid movements and high cloud tops. This will have the additional advantage of being able to provide forecasts for remote communities and over oceans where radar data is not available.

% \footnote{Photo by \href{https://unsplash.com/@wistomsin?utm_source=unsplash&utm_medium=referral&utm_content=creditCopyText}{Tom Barrett} on \href{https://unsplash.com/photos/hgGplX3PFBg?utm_source=unsplash&utm_medium=referral&utm_content=creditCopyText}{Unsplash}
% }

\begin{questions}
    \item Is a sequence to sequence model trained on satellite images capable of forecasting radar composite images in the near future ?
    \item To what extent does other data aid the model in predicting the radar images ?
    \item Which deep learning architecture is the most adequate for this problem ?\label{itm:qwithlabel}
\end{questions}

\section{Main Contribution}
This proposal has the potential to contribute a model capable of predicting near future radar composite images.
provide a way of predicting near future radar composites from satellite images. Investigate architectures that allow the best way to do this.

\section{Related Works}

study \cite{nhess-22-577-2022} measured the importance of features for thunderstorm prediction, it found that satellite was the strongest feature, they suggest that satellite images are also valuable and can offer a good alternative for places with missing satellite data, like over oceans and less developed areas.

Additionally data driven methods have been used to predict precipitation. Additionally various foundation models have been presented to be fine-tuned to solve meteorological tasks \cite{nguyen2023climax, bi2022panguweather}.

\section{Background}

\section{Methods}

\section{Expected Results}

\subsection{Data}

We have obtained a mulit-year structured archive of radar composites at 5-minute intervals. These radar composites are made from 5 doppler radars that provide a good coverage of the benelux area:
\begin{itemize}
    \item Den Helder
    \item Herwijnen
    \item Essen
    \item Borkum
    \item Neuheilenbach
\end{itemize}

\begin{figure}
    \centering
    \includegraphics{images}
    \caption{Caption}
    \label{fig:my_label}
\end{figure}

One radar in den helder and one in herwijnen, possible other soures from germany. \texttt{hdf5} files

\textsc{EUMETSAT} \textsc{SEVIRI} multichannel satellite imagery covering large partst of europe. \texttt{GRIB} files.

\subsection{Implementation Details}

\subsection{Metrics}

\begin{enumerate}
    \item MSE
    \item MAE
    \item RMSE
\end{enumerate}

exteded to perceptual and other content losses originating from superzoom studies.

\subsection{Experiments}


\section{Time Planning}

\medskip

\begin{ganttchart}[hgrid,vgrid,]{1}{12}
  \gantttitle{\textbf{Planning Table}}{12} \\
  \gantttitlelist{1,...,12}{1} \\
  \ganttgroup{Crispdm}{1}{3} \\
  \ganttbar{Proposal}{1}{2} \\
  \ganttlinkedbar{Task 2}{3}{8} \ganttnewline
  \ganttbar{Final Task}{8}{12}
  \ganttlink{elem2}{elem3}
\end{ganttchart}



\section{Conclusions}



\begin{acks}
Infoplaza B.V
\end{acks}

\bibliographystyle{ACM-Reference-Format}
\bibliography{ref}

\appendix
%Appendix A
\section{Headings in Appendices}
The rules about hierarchical headings discussed above for the body
of the article are different in the appendices. In the
\textbf{appendix} environment, the command \textbf{section} is
used to indicate the start of each Appendix, with alphabetic order
designation (i.e. the first is A, the second B, etc.) and a title
(if you include one).  So, if you need hierarchical structure
\textit{within} an Appendix, start with \textbf{subsection} as the
highest level. Here is an outline of the body of this document in
Appendix-appropriate form:
\subsection{Appendix A.1}
Sample text.
\subsection{Appendix A.2}
Sample text.
\subsubsection{Appendix A.2.1}
Sample text.

\paragraph{Inline (In-text) Equations}
% This next section command marks the start of
% Appendix B, and does not continue the present hierarchy
\section{Appendix B}
Sample text.
%\balancecolumns % GM June 2007
% That's all folks!


\end{document}
\endinput
%%
%% End of file `sample-authordraft.tex'.
