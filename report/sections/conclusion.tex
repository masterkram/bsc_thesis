In this paper we investigate possible methods to create a model capable of
predicting precipitation based on satellite images.
We propose and train 3 different model architectures. The trained models are benchmarked and compared based on several metrics.
All models are framed in two different tasks: classification and regression.
In the classification task we train a model to predict from 8 classes mapping to rain intensity categories. In pixel level regression we do not modify the 
radar reflectivity data and we train a pixel-level regression model.
We find that all trained models are capable of predicting some precipitation but they require improvements to perform well in practice.
This can be attributed to relatively low amounts of data being used, difference in resolution
of the two types of data, data imbalances and the complexity of the problem space. 
The model that achieves the best score over the test dataset is the ConvLSTM model. The attention mechanism
as used currently seems to make predictions worse for both classification and regression. More testing needs to be done to state certainly if ConvLSTM surpasses
the performance of 3D U-Net since form a visual standpoint U-Net appears to give more meteorologically viable results.