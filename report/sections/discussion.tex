A longer period is necessary to produce better results.

Model could be underfit, requiring more epochs and resources.

For U-Net the experimental setup is slightly different since we increased the available satellite timesteps from 5 to 8.
This is due to the fact that the encoder expects at least 8 in the depth dimension, to be able to encode the input patch.


A core issue with this problem is data imbalance. There is data imbalance at two levels.
First of all at the level of the datasets training, test and validation have different amounts of rain.
Optimally we would like to have a test dataset that has both stretches of time with large, and low amounts of precipitation.
As of now we have just cut off the datasets based on the available file count and the chosen split ratios.
Another source of data imbalance is the imbalance comes from the images. The radar images are most of the time presenting conditions with no rain.
When there is rain present on the images it is rare that this surpasses 50\% of the image. This makes it difficult to produce suitable models because
the model is biased towards predicting no rain. We have used known methods for imbalanced datasets such as adding class weights to 
the loss function. Other methods such as focal loss exist, which could be investigated in further research.

The problem also presents difficulties because the resolution of the satellite images is worse than the required output.
This could be improved with the high resolution visual channel of the satellite which has a 1km spatial  resolution.

radar and satellite pixels are not perfectly aligned grids.
radar has more spatial resolution 1x1 km while satellite has 3x3 km at the
sub satellite point. At the edge each pixel covers 12km area.