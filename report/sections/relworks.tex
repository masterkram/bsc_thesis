In this section we will discuss the existing work in precipitation nowcasting via machine learning. This section is structured by the type of input data used in the approaches, first we list radar based learning and then we discuss satellite based approaches.

\subsection{Radar Based Nowcasting}

Recurrent neural networks (\textsc{RNNs}) have been created to learn temporal relationships in data, therefore they are a natural candidate to the task of learning spatio-temporal patterns of weather. The \textsc{LSTM} architecture was developed by Hochreiter and Schmidhuber \cite{lstm}, to solve the problem of vanishing and exploding gradients in RNNs and is widely used. Taking LSTM as a base and adapting the weights to kernels, \textsc{ConvLSTM} \cite{convlstm} was introduced for the task of precipitation nowcasting. Multiple layers of ConvLSTM are used in this paper to obtain a sequence to sequence architecture. A further improvement of ConvLSTM is \textsc{trajGRU} which was proposed by Shi et al. \cite{shi2017deep} to be able to learn the \textit{location-variant} structure for recurrent connections.
\smallskip

Pure convolutional neural networks have also been used to predict precipitation. As demonstrated by Bai et al. and Gering et al. \cite{bai2018empirical, gehring2017convolutional} convolutional neural architectures can outperform recurrent neural networks for a variety of sequence modelling tasks. This is the reason why many works on precipitation nowcasting have opted for pure convolutional networks \cite{rainet,agrawal2019machine}.
\smallskip

Due to machine learning models attempting to minimize loss, \textit{blurry predictions} can be produced by models. This can be alleviated by using generative models which sample from the possible futures and do not seek to provide a best average fit. Generative Adversarial Networks have been successfully applied to the task of precipitation nowcasting \cite{Ravuri_2021}.

\subsection{Satellite Based Nowcasting}

In their study Chen et al. built a \cite{precipitationEstimationFromSat} MLP to forecast radar data from satellite data. The researchers used a combination of low earth orbit satellite passive microwave and infrared channels from two different satellites. Their model is developed to predict up to 1.5 hours in the future by recursive predictions of the model.
\smallskip

A study that does not predict precipitation but uses lightning as a marker for extreme precipitation was performed by Brodehl et al. \cite{predictionLightning} this study uses a convolutional network to predict lightning events, and contributes the important observation that both the visual and infrared channels are important in differing ways to predict lightning.
\smallskip

The approach taken by the researchers of MetNet \cite{sønderby2020metnet} is to combine a convolutional block for spatial downsampling, then a ConvLSTM block for temporal encoding and finally a Axial attention block \cite{vaswani2017attention}. MetNet is able to perform more accurate forecasts than NWP models for up to 8 hours. In this study the input data that is used is both satellite and radar data as well as the elevation, time of the year and latitude and longitude values.